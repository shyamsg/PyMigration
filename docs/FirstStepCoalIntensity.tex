\documentclass[11pt]{article}
\usepackage{geometry}                % See geometry.pdf to learn the layout options. There are lots.
\geometry{letterpaper}                   % ... or a4paper or a5paper or ... 
%\geometry{landscape}                % Activate for for rotated page geometry
%\usepackage[parfill]{parskip}    % Activate to begin paragraphs with an empty line rather than an indent
\usepackage{graphicx}
\usepackage{amssymb}
\usepackage{epstopdf}
\DeclareGraphicsRule{.tif}{png}{.png}{`convert #1 `dirname #1`/`basename #1 .tif`.png}

\title{Computing coalescent intensities for an island model with varying population sizes and migration rates}
\author{Shyam Gopalakrishnan}
%\date{}                                           % Activate to display a given date or no date

\begin{document}
\maketitle
\section{Island model with $K$ demes}
Consider two lineages in an island model with $K$ demes. Let $N_i$ be the population
size for the $i$-th deme and $m_{ij}$ be the migration rate from deme $i$ to deme
$j$. Additionally, let the migration rates be symmetric, i.e., $m_{ij} = m_{ji}$. 

\section{Computing coalescent intensities}
Let $P_{ij}^{t}$ be the probability that two lines, one currently residing in deme $i$
and the other in deme $j$, coalesce within $t$ generations. For two lines sampled 
from the same deme $i$, this probability is written as $P_{ii}^{t}$.\\
Using a first step analysis, we can set up a regression to compute these coalescent 
probabilities. 
\begin{eqnarray}
P_{ii}^t &=& 1\cdot P(C_i) + \sum_{k=1, k \neq i}^{K}P(i \rightarrow k) P_{ik}^{t-1} + \nonumber \\
              &  & (1-P(C_i) -\sum_{k=1, k \neq i}^K P(i \rightarrow k)) P_{ii}^{t-1} \\
P_{ij}^t &=& P(i\rightarrow j) P_{jj}^{t-1} + P(j\rightarrow i) P_{ii}^{t-1} + \sum_{k=1,k\neq i,j}^K (P(i\rightarrow k)P_{kj}^{t-1}+P(j\rightarrow k)P_{ik}^{t-1}) \nonumber \\
              &  & + (1-\sum_{k=1,k\neq i,j}^K (P(i\rightarrow k) + P(j\rightarrow k)) - P(i\rightarrow j) - P(j\rightarrow i))P_{ij}^{t-1}
\end{eqnarray}
Here, $i\rightarrow j$ signifies a migration from deme $i$ to $j$ and $C_i$ signifies a coalescent event between the two lines when both reside in 
deme $i$. The migration and coalescent probabilities are given below.
\begin{eqnarray}
P(C_i) &=& \frac{1}{2N_i} \\
P(i\rightarrow j) &=& n_im_{ij}
\end{eqnarray}
Noting that $n_i=2$ when both lines are in the same deme and $n_i=n_j=1$ when one line resides in demes $i$ and $j$ each, and substituting the above values into the recursion equations we get,
\begin{eqnarray}
P_{ii}^t &=& \frac{1}{2N_i} + 2\sum_{k=1, k \neq i}^{K} m_{ik}P_{ik}^{t-1} + (1-\frac{1}{2N_i}-2\sum_{k=1, k \neq i}^K m_{ik}) P_{ii}^{t-1} \\
P_{ij}^t &=& m_{ij}(P_{jj}^{t-1}+P_{ii}^{t-1})+\sum_{k=1,k\neq i,j}^K (m_{ik}P_{kj}^{t-1}+m_{jk}P_{ik}^{t-1})+ \nonumber \\
         & & (1-2m_{ij}-\sum_{k=1,k\neq i,j}^K (m_{ik}+m_{jk}))P_{ij}^{t-1}
\end{eqnarray}
The above probabilities can be computed, all at the same time, using a dynamic programming approach. The boundary conditions for these recursion
equations are given at t=0, when $P_{ii}^0 = P_{ij}^0 = 0$.\\
We can modify these equations to pose this recursion as a matrix recursion. 
\begin{equation}
P^t = C + MP^{t-1} + (MP^{t-1})^T = C + MP^{t-1} + P^{t-1}M 
\end{equation}
Here $C$ is a diagonal matrix with $C_{ii} = 1/2N_i$, $M$ is the \textit{symmetric} migration matrix with diagonal elements $M_{ii} = 0.5 - \sum_{j=1, j \neq i}^K M_{ij}$ and
$P^{t-1}$ is the coalescent intensity matrix at $t-1$ generations. Adding a couple of steps to the previous equation, we get the two and three step recursion as given below.
\begin{eqnarray}
P^t &=& C + MC + M^2P^{t-2} + MP^{t-2}M + CM + MP^{t-2}M + P^{t-2}M^2 \nonumber \\
       &=& C + MC + CM + M^2P^{t-2} + 2MP^{t-2}M + P^{t-2}M^2 \\
P^t &=& C + MC +CM + M^2C + M^3P^{t-3} + M^2P^{t-3}M + 2MCM +\nonumber \\
       &  &  2M^2PM+ 2MPM^2 + CM^2+MP^{t-3}M^2+P^{t-3}M^3 \nonumber \\
       &=& C + MC + CM + M^2C + 2MCM + CM^2 + \nonumber \\
       &  &  M^3P^{t-3} + 3M^2P^{t-3}M + 3MP^{t-3}M^2 + P^{t-3}M^3
\end{eqnarray}

\end{document}  
